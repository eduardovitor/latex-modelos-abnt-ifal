% ------------------------------------------------------------------------------
% Dados do trabalho acadêmico
% ------------------------------------------------------------------------------

\titulo{Título do Trabalho}
%\title{Title in English}
\subtitulo{Subtítulo do trabalho}
\autor{Nome completo do autor}
\local{Alagoas}
\data{Outubro de 2021} % Normalmente se usa apenas mês e ano

% ------------------------------------------------------------------------------
% Natureza do trabalho acadêmico
% Use apenas uma das opções:
% - Tese (para Doutorado)
% - Dissertação (para Mestrado)
% - Projeto de Qualificação (para Mestrado ou Doutorado)
% - Trabalho de Conclusão de Curso (para Graduação)
% ------------------------------------------------------------------------------

\projeto{Projeto de Qualificação}

% ------------------------------------------------------------------------------
% Título acadêmico
% Use apenas uma das opções:
% - Doutor (para Doutorado)
% - Mestre (para Mestrado)
% - Bacharel (para Graduação)
% ------------------------------------------------------------------------------

\tituloAcademico{Doutor}

% ------------------------------------------------------------------------------
% Dados da instituição
% ------------------------------------------------------------------------------

\instituicao{INSTITUTO FEDERAL DE ALAGOAS - CAMPUS ARAPIRACA}
\programa{BACHARELADO EM SISTEMAS DE INFORMAÇÃO}

% ------------------------------------------------------------------------------
% Área de concentração e linha de pesquisa
% Observação: Indique o nome da área de concentração e da linha de pesquisa do
% programa de Pós-graduação nas quais este trabalho se insere. Se a natureza
% for Trabalho de Conclusão de Curso, deixe ambos os campos vazios.
% ------------------------------------------------------------------------------

%\areaconcentracao{Modelagem Matemática e Computacional.}
%\linhapesquisa{Métodos Matemáticos Aplicados.}

% ------------------------------------------------------------------------------
% Logomarca
% Observação: A logomarca da instituição deve ser colocada no mesmo diretório
% onde foi colocado o documento principal.
% O formato pode ser: pdf, eps, jpg ou png.
% ------------------------------------------------------------------------------

\logoinstituicao{3cm}{figuras/IF_vertical.png}

% ------------------------------------------------------------------------------
% Dados do(s) orientador(es)
% ------------------------------------------------------------------------------

\orientador{Prof. Dr. Nome do orientador}
%\orientador[Orientadora:]{Nome da orientadora}
\instOrientador{Instituição do orientador}

\coorientador{Prof. Dr. Nome do coorientador}
%\coorientador[Coorientadora:]{Nome da coorientadora}
\instCoorientador{Instituição do coorientador}

% ------------------------------------------------------------------------------
% Folha de Rosto
% ------------------------------------------------------------------------------

% Trabalho de Conclusão de Curso
%\preambulo{{\imprimirprojeto} apresentado ao Curso de Engenharia de Computação do Centro Federal de Educação Tecnológica de Minas Gerais, como requisito parcial para a obtenção do título de {\imprimirtituloAcademico} em Engenharia de Computação.}

% Projeto de qualificação de Mestrado ou Doutorado
\preambulo{{\imprimirprojeto} apresentado ao Programa de \mbox{Pós-graduação} em Modelagem Matemática e Computacional do Centro Federal de Educação Tecnológica de Minas Gerais, como requisito parcial para a obtenção do título de {\imprimirtituloAcademico} em Modelagem Matemática e Computacional.}

% Dissertação de Mestrado
%\preambulo{{\imprimirprojeto} apresentada ao Programa de \mbox{Pós-graduação} em Modelagem Matemática e Computacional do Centro Federal de Educação Tecnológica de Minas Gerais, como requisito parcial para a obtenção do título de {\imprimirtituloAcademico} em Modelagem Matemática e Computacional.}

% Tese de Doutorado
%\preambulo{{\imprimirprojeto} apresentada ao Programa de \mbox{Pós-graduação} em Modelagem Matemática e Computacional do Centro Federal de Educação Tecnológica de Minas Gerais, como requisito parcial para a obtenção do título de {\imprimirtituloAcademico} em Modelagem Matemática e Computacional.}
